\documentclass[a4paper]{article}

\usepackage[colorlinks=false, linktocpage=true]{hyperref}
\usepackage[T1]{fontenc}
\usepackage{lmodern}
\usepackage{tikz}
\usepackage{listings}
\usepackage{capt-of}
\usepackage{graphicx}
\usepackage{deluxetable}


\newcommand{\flame}{\texttt{flame}}
\newcommand{\fuel}{\texttt{fuel}}
\newcommand{\inp}{\texttt{input}}


%\lstset{language=IDL}

% define the font for file names
\newenvironment{filenamefont}{\fontfamily{pcr}\selectfont}{\par}
\DeclareTextFontCommand{\filename}{\filenamefont}

% define the font for settings and structures
\newenvironment{codefont}{\fontfamily{cmtt}\selectfont}{\par}
\DeclareTextFontCommand{\code}{\codefont}

% Title page -------------------------------------------------------------------
% code taken from https://tex.stackexchange.com/questions/85904

\definecolor{titlepagecolor}{cmyk}{1,.60,0,.40}

\DeclareFixedFont{\titlefont}{T1}{ppl}{b}{}{0.5in}

\makeatletter
\def\printauthor{%
    {\Large \@author}}
\makeatother
\author{%
\textsc{Sirio Belli}
\texttt{sirio@mpe.mpg.de}
    }

\newcommand\titlepagedecoration{%
\begin{tikzpicture}[remember picture,overlay,shorten >= -10pt]

\coordinate (aux1) at ([yshift=-15pt]current page.north east);
\coordinate (aux2) at ([yshift=-410pt]current page.north east);
\coordinate (aux3) at ([xshift=-4.5cm]current page.north east);
\coordinate (aux4) at ([yshift=-150pt]current page.north east);

\begin{scope}[titlepagecolor!40,line width=12pt,rounded corners=12pt]
\draw
  (aux1) -- coordinate (a)
  ++(225:5) --
  ++(-45:5.1) coordinate (b);
\draw[shorten <= -10pt]
  (aux3) --
  (a) --
  (aux1);
\draw[opacity=0.6,titlepagecolor,shorten <= -10pt]
  (b) --
  ++(225:2.2) --
  ++(-45:2.2);
\end{scope}
\draw[titlepagecolor,line width=8pt,rounded corners=8pt,shorten <= -10pt]
  (aux4) --
  ++(225:0.8) --
  ++(-45:0.8);
\begin{scope}[titlepagecolor!70,line width=6pt,rounded corners=8pt]
\draw[shorten <= -10pt]
  (aux2) --
  ++(225:3) coordinate[pos=0.45] (c) --
  ++(-45:3.1);
\draw
  (aux2) --
  (c) --
  ++(135:2.5) --
  ++(45:2.5) --
  ++(-45:2.5) coordinate[pos=0.3] (d);
\draw
  (d) -- +(45:1);
\end{scope}
\end{tikzpicture}%
}

\begin{document}

\begin{titlepage}

\vspace*{2cm}
\noindent
\Huge \textbf{\textit{Flame} user manual}\par
\vspace*{1cm}
\noindent
\Large Last updated \today \\
\null\vfill
\vspace*{1cm}
\noindent
\hfill
\begin{minipage}{0.35\linewidth}
    \begin{flushright}
        \printauthor
    \end{flushright}
\end{minipage}
%
\begin{minipage}{0.02\linewidth}
    \rule{1pt}{50pt}
\end{minipage}
\titlepagedecoration
\end{titlepage}

% end title page -------------------------------------------------------------------

\newpage
\tableofcontents

% ***********************************************************************************************
% ***********************************************************************************************
%									1. INTRODUCTION
% ***********************************************************************************************
% ***********************************************************************************************

\newpage
\section{Introduction}
\label{sec:introduction}

\flame\ is a pipeline for reducing near-infrared multi-object spectroscopic data, written in IDL. The pipeline design and the algorithms used are described in Belli, Contursi \& Davies (2017), which users should read before reading the present manual. Please cite that paper in your work if you made use of \flame.

Although created specifically for the LUCI instrument at the Large Binocular Telescope, \flame\ has been designed in a modular way and can be easily adapted to work with the data produced by different instruments. For simplicity, the present document describes in detail the reduction of LUCI data. Section \ref{sec:instruments} explains how to use \flame\ with other supported instruments, and how it is possible to implement support for new instruments.

For comments, bug reports, or suggestions please contact \texttt{sirio@mpe.mpg.de}.



% ***********************************************************************************************
% ***********************************************************************************************
%									2. INSTALLATION
% ***********************************************************************************************
% ***********************************************************************************************

%%%
%%% Note: the links do not show up on the pdf, even though they work if you know where to click.
%%%

\section{Installation}
\label{sec:installation}


\subsection{Requirements}

The pipeline has been extensively tested in Linux and Mac. Limited testing was done in Windows.

The following three external IDL libraries must be installed before running \flame.
\begin{enumerate}
\item NASA's \href{http://idlastro.gsfc.nasa.gov/}{IDL Astronomy User's Library};
\item David Fanning's \href{http://www.idlcoyote.com/code_tips/installcoyote.php}{Coyote Library};
\item Craig Markwardt's \href{https://www.physics.wisc.edu/~craigm/idl/fitting.html}{mpfit} (only the mpfit.pro and mpfit2dfun.pro files are needed).
\end{enumerate}
Make sure to have the most recent versions of these libraries (at least more recent than the January 2016 release).

Other third-party routines used by \flame\ are included in the distribution, in the \texttt{flame/lib/} directory. These are the B-spline routines from the \texttt{idlutils} library, the L.A.Cosmic routine written by Joshua Bloom and Pieter van Dokkum, and the readmhdufits routine written by Marc Kassis.


\subsection{Download}

Download \flame\ from \url{http://siriobelli.github.io/flame/}. You can either save it in the directory where you keep your IDL code, or save it somewhere else and then add the \texttt{flame/} directory to the IDL path. No other step is necessary.





% ***********************************************************************************************
% ***********************************************************************************************
%									3. SETUP
% ***********************************************************************************************
% ***********************************************************************************************



\section{Setting up the Data Reduction}
\label{sec:setup}


\subsection{Basic input}

Go to the directory that you want to use for the reduction of a data set (it is recommended to create a new directory for this purpose). Then:
\begin{itemize}
\item Copy the appropriate driver file (for LUCI, \texttt{flame\_driver\_luci.pro}) from the \texttt{flame/drivers/} directory to the current directory. If you want to see all the advanced options, use the extended driver file instead\\ (\texttt{flame\_driver\_complete.pro}).
\item Create a text file (e.g., \texttt{science.txt}) with, in each line, the full name, including the absolute path, of a raw science frame (either FITS files or gzipped FITS files).
\item \emph{(optional)} Create additional text files with the names of the calibration frames (e.g., \texttt{darks.txt}, \texttt{flats.txt}, etc.)
\end{itemize}

The data reduction is controlled entirely via the driver file. Open the local copy with a text editor and have a look at the code. First the input parameters are set, the pipeline is then initialized, and finally the actual data reduction is run.

The first step in the driver file creates the \inp\ structure:
\begin{lstlisting}
 input = flame_create_input()
\end{lstlisting}
We now need to populate the fields in the \inp\ structure with values that are appropriate for the data set that we are reducing.

\subsubsection{Science frames}
We start by specifying the name of the file containing the science frames; for example:
\begin{lstlisting}
input.science_filelist = 'science.txt'
\end{lstlisting}
If the file is not in the current working directory, then the absolute path must be provided.

\subsubsection{Nodding}
Near-infrared observations typically use spatial nodding for an effective sky subtraction. \flame\ supports a two-point nodding, also called A-B nodding, which is activated via the flag:
\begin{lstlisting}
input.AB_subtraction = 1
\end{lstlisting}

\subsubsection{Reference star}
It is always recommended to place a slit on a relatively bright star (hereafter called the reference star) from which the observing conditions and the spatial dithering can be accurately measured. In the A-B nodding scheme, the star trace will appear on two different positions, which may be in the same slit or in two different slits. The y pixel position of the reference star in both the A and the B position must be provided:
\begin{lstlisting}
input.star_y_A = 520
input.star_y_B = 560
\end{lstlisting}
These coordinates do not need to be exact. If the trace of the reference star is very faint in a single frame, a simple A-B subtraction is generally sufficient for a visual detection. When no reference star is available, the nodding pattern must be specified manually (see Section \ref{sec:manual_dithering}). If no AB nodding is performed, the B star position should be set to zero.

\subsubsection{Slits to be reduced}
While the default behavior is to reduce all the slits in the mask, it is sometimes preferable to reduce only one slit, for example to obtain the reduction of one representative object while observing. This can be achieved by specifying:
\begin{lstlisting}
input.reduce_only_oneslit = 2
\end{lstlisting}
where in this case we would get the reduction of only the second slit. The default value is zero which means that all the slits will be reduced.

\subsubsection{Longslit observations}
Finally, to reduce longslit observations, it is sufficient to set the longslit flag:
\begin{lstlisting}
input.longslit = 1
\end{lstlisting}
Because the longslit typically spans the whole detector in the vertical range, the edges of the longslit are not well defined. The region of interest must be specified by providing the y pixel coordinates of the top and bottom limits:
\begin{lstlisting}
input.longslit_edge = [1050, 1350]
\end{lstlisting}
If longslit\_edge is set to [0,0], the default is to reduce the full frame excluding 10\% of pixels at the bottom and 10\% at the top. For example, for a frame that is 2048 pixels tall, this would correspond to the [205, 1843] range. Since the slit curvature cannot be traced, the spatial distortion will not be rectified for longslit data.


\subsection{Optional input}

The \inp\ structure contains additional fields that can be used to specify optional inputs. The full list is reported in Table \ref{tab:input}, and the content of the \inp\ structure can be explored at any time by typing
\begin{lstlisting}
help, input
\end{lstlisting}
in the interactive IDL session.

The extended driver file \texttt{flame\_driver\_complete.pro} includes all the possible types of input and settings: the relevant lines can be copied and pasted in the local copy of the driver.

% ******************************************************************************
\begin{table}
\caption{Content of the \inp\ structure \label{tab:input}}
\medskip
\footnotesize
\makebox[\textwidth][c]{
  \begin{tabular}{lll}
    \hline
    \smallskip
    Field    & Default Value   & Description \\
    \hline
    \smallskip
science\_filelist   	     & 'science.txt' 		& name of ASCII file containing the list of science frames \\
AB\_subtraction      & 0      & flag to enable AB subtraction \\
star\_y\_A 		 & 0			& pixel y-coordinate of the A position of the star trace \\
star\_y\_B 		 & 0			& pixel y-coordinate of the B position of the star trace \\
reduce\_only\_oneslit 	& 0 				& if non-zero, reduce only that one slit \\
longslit 				& 0 				& set to one to reduce long-slit data \\
longslit\_edge			& [0, 0] 			& when reducing long-slit data, only consider this range of pixel y-coordinate \\
dark\_filelist   		       & 'none' 				& name of ASCII file containing the list of dark frames\\
pixelflat\_filelist        & 'none' 				& name of ASCII file containing the list of frames for pixel flat field	\\
illumflat\_filelist   		 & 'none' 				& name of ASCII file containing the list of frames for illumination flat field	\\
illumflat\_pixelshift  		 & 0      				& vertical pixel shift required to align the illumination flat with the science frames	\\
slitflat\_filelist    & 'none'        & name of ASCII file containing the list of slit flats \\
slitflat\_pixelshift  & 0      				& vertical pixel shift required to align the slit flat with the science frames	\\
arc\_filelist   		  & 'none' 				& name of ASCII file containing the list of lamp arc frames	\\
arc\_pixelshift  		  & 0      				& vertical pixel shift required to align the arc with the science frames	\\
slit\_position\_file  & 'none'        & name of ASCII file containing the approximate y-coordinates of the slit edges \\
dither\_file       		& 'none' 				& name of ASCII file containing the list of dither positions \\
max\_slitwidth\_arcsec & 0          & threshold value of the slit width to identify the alignment boxes \\
intermediate\_dir		 & 'intermediate/'		& directory that will contain the intermediate products \\
output\_dir				   & 'output/'			& directory that will contain the final output \\
\hline
\end{tabular}
}
\end{table}
% ******************************************************************************


\subsubsection{Calibration frames}
The calibration frames used by \flame\ are the dark, pixel flat, illumination flat, slit flat, and arcs. Each of these types of calibrations can be supplied to the pipeline via text files containing a list of FITS files, similarly to what done for the science data. The only difference is that all the calibration files are optional, and the default value for the filelist parameter is always \texttt{'none'}. Here is an example:
\begin{lstlisting}
input.dark_filelist = 'darks.txt'
input.pixelflat_filelist = 'pixelflats.txt'
input.illumflat_filelist = 'none'
input.slitflat_filelist = 'twilightflats.txt'
input.arc_filelist = 'none'
\end{lstlisting}
For the illumination flat, slit flat, and arcs, it is important that the calibration frames are spatially aligned with the science frames. However, it is often the case that calibrations taken in the afternoon show a vertical displacement compared to science observations because of flexure. In this case, it is possible to correct for this by specifying the vertical offset, in pixel, that must be applied to the calibration frames:
\begin{lstlisting}
input.illumflat_pixelshift = 7
input.slitflat_pixelshift = 8
input.arc_pixelshift = 0
\end{lstlisting}

\textbf{Warning:} the use of arcs with LUCI observations has not been tested.


\subsubsection{Slit edge positions}
Normally, the expected positions of the slits on the detector are derived from the values in the FITS header. However, sometimes this can fail, particularly when the slits are tilted or curved. In these cases it is possible to specify the list of approximate pixel positions of each slit edge as an ASCII file:
\begin{lstlisting}
input.slit_position_file = 'slit_edges.reg'
\end{lstlisting}
The easiest way to generate this file is to open the slit flat field (or a science frame) with ds9, make a simple circular region at each slit edge (both top and bottom edges for each slit must be specified), and save the regions in a file with format 'XY' and coordinate system 'Image'.



\subsubsection{Dithering positions}
\label{sec:manual_dithering}
It is advisable to always have a reference star on the multi-slit mask. When this is not possible, \flame\ cannot measure the nodding and dithering offset for each frame, and this pattern needs to be saved in an ASCII file. The file must contain a number of lines equal to the number of science frames. Each line will have the value of the vertical offset, in arcseconds, corresponding to that frame (for example 5, -5, 5, -5, etc). Then the file name needs to be input in the \inp\ structure:
\begin{lstlisting}
input.dither_file = 'dither.txt'
\end{lstlisting}

\subsubsection{Directories}
Finally, it is possible to specify the directories for intermediate and final outputs:
\begin{lstlisting}
input.intermediate_dir = 'intermediate/'
input.output_dir = 'output/'
\end{lstlisting}
If these directories do not exist, they will be created by \flame. Note that these definitions use relative paths, but absolute paths can also be used.



\subsection{Initialization}

The data reduction in \flame\ consists of a series of steps, which are represented by individual routines. The information is carried from one routine to the next via the use of only one structure, named \fuel, which contains all the input parameters set by the user, together with all the internal variables.

Once the \inp\ fields have been specified, the \fuel\ structure must be created:
\begin{lstlisting}
fuel = flame_initialize_luci(input)
\end{lstlisting}
This will also check for potential errors in the inputs, and create the necessary directories.

The \fuel\ structure consists of six substructures:
\begin{itemize}
	\item fuel.input: a copy of the \inp\ structure created in the previous step;
  \item fuel.settings: a series of settings and parameters that control the data reduction;
	\item fuel.util: utility variables needed by \flame;
	\item fuel.instrument: instrument-specific information;
	\item fuel.diagnostics: diagnostics such as seeing and transparency for each of the science frames;
	\item fuel.slits: information relative to each slit, such as position on the detector and expected wavelength range.
\end{itemize}

Some of the fields in the \fuel\ structure are initially set to null pointers and will be created by successive steps.

During the initialization, relevant fields from the FITS header of the science files, such as band, central wavelength, and slit positions are read and saved. This is the only step during the data reduction where instrument-specific operations are performed.

After the initialization, the user has complete control over the \texttt{fuel.settings} structure. We will describe some of the settings in the next section; the complete list together with a brief description for each of them is shown in Table \ref{tab:settings}. Normally, the user should not edit the other five substructures contained in \fuel.

% ******************************************************************************
\begin{table}
\caption{Content of the \texttt{fuel.settings} structure \label{tab:settings}}
\medskip
\footnotesize
  \makebox[\textwidth][c]{
  \begin{tabular}{llll}
    \hline
    \smallskip
    Field    & Default Value   & Routine   & Description \\
             & (for LUCI)      &           & \\
    \hline
    \smallskip
    sky\_emission\_filename       & -             & roughwavecal     & None \\
    linelist\_filename            & -             & (many)           & None \\
    star\_x\_range 			          & [1000, 1200]  & diagnostics   	 & range of pixel x-coordinate to consider for the reference star trace \\
    star\_y\_window               & $\simeq$4 arcsec & diagnostics   & range of pixel y-coordinate to consider for the reference star trace \\
    clean\_individual\_frames     & 0             & calibrations     & identify and mask cosmic rays in each frame \\
    badpix\_sigma                 & 7.0           & calibrations     & sigma clipping to identify bad pixels in the master dark frame \\
    badpix\_flatcorrection        & 0.20          & calibrations     & values beyond this deviation in the master pixel flat are bad pixels \\
    trace\_slit\_with\_emlines    & 1             & slitid           & use arcs or sky lines to trace slit edges \\
    trace\_slit\_ymargin          & 12            & slitid           & margin (in pixels) to consider beyond the expected slit edge \\
    trace\_slit\_polydegree       & 2             & slitid           & degree of the polynomial used to describe the slit edge \\
    roughwavecal\_R               & [500, 1000, 3000] & roughwavecal & spectral resolution to be used for the rough wavelength calibration \\
    roughwavecal\_smooth\_window  & 20            & roughwavecal     & window, in pixels, used to calculate and subtract the sky continuum \\
    roughwavecal\_split           & 0             & roughwavecal     & in the last step of rough\_wavecal, split the wavelength range in two \\
    findlines\_stack\_rows        & 0             & findlines        & number of pixel rows to stack when identifying arcs or sky lines \\
    findlines\_poly\_degree       & 5             & findlines        & polynomial degree for the row-by-row wavelength solution \\
    findlines\_Nmin\_lines        & 6             & findlines        & minimum number of lines needed to consider a pixel row \\
    findlines\_linefit\_window    & 6.0           & findlines        & window used for Gaussian fitting, in units of expected line width \\
    wavesolution\_order\_x        & 3             & wavecal          & polynomial degree for the wavelength solution, along x \\
    wavesolution\_order\_y        & 2             & wavecal          & polynomial degree for the wavelength solution, along y \\
    shift\_arcs\_Nmin\_lines      & 1             & wavecal          & minimum number of sky lines needed for shifting the arcs solution \\
    illumination\_correction      & 1             & illumcorr        & apply illumination correction derived from sky lines \\
    skysub\_plot\_range           & [0.4, 0.6]    & skysub           & fractional range of the x axis to plot during sky subtraction \\
    skysub\_bspline\_oversample   & 1.0           & skysub           & breakpoints separation for the b-spline model, in units of pixels \\
    skysub\_reject\_fraction      & 0.10          & skysub           & fraction of pixels to reject at each loop \\
    skysub\_reject\_loops         & 3             & skysub           & number of loops for outlier rejection \\
    skysub\_reject\_window        & 2.0           & skysub           & window for outlier rejection, in units of breakpoint separation \\
    interpolation\_method         & 'NaturalNeighbor'  & rectify     & method used to resample frames during rectification \\
    frame\_weights                & 'None'        & combine          & weights to use for the final frmae stacking \\
    combine\_sigma\_clip          & 2.0           & combine          & sigma-clipping to use when combining frames \\
    combine\_min\_framefrac       & 0.49          & combine          & minimum fraction of frames that must contribute to a pixel \\
    stop\_on\_error               & 1             & (all)            & halt execution when an error occurs \\
    \hline
  \end{tabular}
  }
\end{table}
% ******************************************************************************




% ***********************************************************************************************
% ***********************************************************************************************
%									4. DATA REDUCTION
% ***********************************************************************************************
% ***********************************************************************************************



\section{Running the Data Reduction}
\label{sec:data_reduction}

Once the first part of the driver file has been edited, the data reduction can be executed. Broadly speaking there are two ways to run \flame:
\begin{enumerate}
\item The user can copy line-by-line the content of the driver file into an interactive IDL session. This allows the user to check the result of each individual step, exploring the \fuel\ structure and the output files and plots.

\item The driver can be executed automatically, by typing:
\begin{lstlisting}
.run flame_driver_luci.pro
\end{lstlisting}
in the IDL command line (assuming the driver file is located in the current directory). The \code{.run} command executes a file as an IDL \emph{main program}, which means that the driver file must have an END statement but does not need a PRO statement at the beginning, as opposed to regular routines\footnote{Technically, the driver file can also be executed as a \emph{batch file}, by running \code{idl flame\_driver\_luci.pro} at the UNIX command line, although this will produce an error message when the END statement is encountered. This method is discouraged because if one of the modules produces an error, the execution will not stop but will continue with the remaining modules, making it difficult for the user to understand the source of the problem.}.
\end{enumerate}

The driver file contains fourteen steps or modules. Every module is an IDL routine whose name starts with \code{flame\_} and which is saved in a .pro file with the same name in the directory \filename{flame/pro/}. All modules accept (and require) only one argument, which is the \fuel\ structure, so that the IDL code for each module looks the same:
\begin{lstlisting}
flame_modulename, fuel
\end{lstlisting}

Every module edits the fuel structure and/or outputs files in the intermediate directory. Only the last two modules, which create the final results, save the output files in the output directory.

In this section for each module we briefly describe its goal, the output produced, and how to solve potential problems encountered while running \flame.



% ---------------------------------------------------
\subsection{\texttt{flame\_diagnostics}}
\label{sec:diagnostics}

The first step produces important diagnostics which will be used by other modules. These diagnostics can also be helpful to the user to assess the quality of the observations, and can be run in real time while observing, ideally after every new frame or frame pair has been taken.

\begin{figure}[htbp]
\centering
\includegraphics[width=0.9\textwidth]{diagnostics}
\caption{Example of diagnostic plots. From the top: Flux of the star trace, normalized by the median value; FWHM in arcsec; vertical position of the trace in A frames; vertical position of the trace in the B frames; airmass value read from the FITS header.}
\label{fig:diagnostics}
\end{figure}

Starting with the y coordinates specified by the user in \code{fuel.startrace\_y\_pos}, this module detects the reference star position on each frame, and identifies which frames belong to the $A$ and which to the $B$ position (and assigns an $X$ position when no star trace is found). Then, for each frame a Gaussian profile is fit to the star trace, from which the flux, vertical position and FWHM are measured. In addition, the airmass value is read from the FITS header of each frame.

All these quantities are saved in the substructure \code{fuel.diagnostics} and plotted as a function of the frame number in the file \filename{diagnostics.ps}. One example of such file is shown in Figure \ref{fig:diagnostics}. Because of the dithering, the vertical position is shown in different panels for the $A$ and the $B$ frames.

The diagnostic plots can be used to judge the trend of transmission (including the effect of cloud cover), seeing, and drift. In addition to help the user make informed decisions while observing, these measurements are also useful to identify the bad frames that should be excluded from the final data reduction.

\medskip
\noindent
\textbf{Outputs.} The main output file is \filename{diagnostics.ps}. The same information is also available in tabular form in \filename{diagnostics.txt}. It is a good idea to check that each frame has been assigned to the correct nodding position, and that the measured seeing, flux variation, and position are reasonable. If one or more frames have unexpected values, additional plots written to \filename{startrace\_identify\_AB.ps} and \filename{startraces.ps} should be checked.

\medskip
\noindent
\textbf{Troubleshooting.} If some of the fits to the stellar trace fail, there are three things that the user can do to improve the fits:
\begin{enumerate}
\item The input values (\code{input.star\_y\_A} and \code{input.star\_y\_B}) can be slightly adjusted. Particularly if the first frame was used as a reference, and there is substantial drift among frames, choosing a value that is more representative of all frames can help.
\item The stellar trace is extracted from a relatively narrow range of $x$ pixels, which can be changed via \texttt{settings.star\_x\_range}. Sometimes the stellar spectrum only covers part of the detector, for example if the slit was placed at the left or right edge of the mask. If sky lines are strong, it is advisable to choose a relatively clear, narrow range; if the star is very faint, it may be better to use a wider range.
\item The vertical window used to extract the stellar profile can also be changed, via \texttt{settings.star\_y\_window}. The stellar trace should be within this window in all frames, so a wider window should be used when dealing with large drift or dithering. Sometimes fits can fail because the spatial profile is disturbed by the presence of the slit edge; in these cases the size of the window should be decreased.
\end{enumerate}


% ---------------------------------------------------
\subsection{\texttt{flame\_quickstack}}

The second step produces a FITS file which is the difference of a stack of the $A$ frames and a stack of the $B$ frames (as defined in the previous module). This step is also useful while observing, since the simple $A-B$ subtraction often yields a decent sky removal which allows the detection of faint continuum traces or emission lines.

The sole goal of this module is to allow a quick look at the data. This step can be skipped with no consequences for the remaining parts of the data reduction.

This step and the previous ones can be run on virtually any data set without any knowledge on the instrument except for the spatial pixel scale. This minimal version of \flame, designed for on-the-fly use, is described in Section \ref{sec:minimal}.

\medskip
\noindent
\textbf{Outputs.} The only output is the file named \filename{quickstack\_A-B.fits} or similar (according to the nodding pattern).


% ---------------------------------------------------
\subsection{\texttt{flame\_calibrations}}

This module applies a number of calibrations to the raw science frames. First, the bad pixel mask is generated from the input dark and/or flat frames. If neither dark nor flat field frames were supplied, then the default bad pixel mask, stored in flame/data/LUCI, is used.

The user can choose whether individual frames must be cleaned of cosmic rays using L.A.Cosmic (this is generally not needed for LUCI data, as long as the number of frames is enough for sigma-clipping), and the parameters used to identify bad pixels from the master dark and flat field.

The corrected science frames are saved in the intermediate/frames/ directory. Each FITS file contains a second extension with the error spectrum, which is calculated as the combination of read-out noise and Poisson noise.


\begin{figure}[htbp]
\centering
\includegraphics[width=0.7\textwidth]{slits}
\caption{An example of LUCI raw frame, with the regions marking the slits identified by \flame\ shown in green. The slits that are not marked are the boxes for the alignment stars, which are ignored by the pipeline.}
\label{fig:slits}
\end{figure}

% ---------------------------------------------------
\subsection{\texttt{flame\_slitid}}
\label{sec:slitid}

The module \texttt{flame\_slitid} is responsible for detecting the slits from the multi-object slitmask. If slit flat field frames are not supplied, the central three science frames will be stacked and used to generate the master slit flat. The vertical coordinates of the slit edges calculated during the initialization are used. However, the physical position of the mask with respect to the detector is typically difficult to predict and varies slightly from one night to the other. To account for this, a vertical shift is calculated by identifying the slit edges.

Once the approximate pixel position of each slit has been calculated, a second, more refined slit edge detection is run. This is based on the identification of individual OH emission lines, which are typically very bright and can be reliably used to trace the edge of a slit. For each slit, a low-order polynomial is fit to the $y$ coordinate of the edge as a function of $x$ position, and the results are saved to the fuel structure.

In some cases, particularly in the $K$ band, there are not enough bright OH lines for identifying the slit edges. It is possible to use the continuum emission instead, by specifying settings.trace\_slit\_with\_emlines$ = 0$. Using lamp flats or arc frames as slit flat field is also an option. The user can also change the order of the polynomial describing the slit edge and the cutout size used for the slit identification.

This module creates the file file slits.reg, which is a ds9 region file that can be loaded on top of any science frame and will show the slit edges (see Figure \ref{fig:slits}). This can be used to check the slit detection and also to identify which slit on the science frame corresponds to which target. The raw detection of the sky lines are saved in slits\_raw.reg.

If the slit identification fails or is not accurate, one alternative is to use the longslit option to manually select the region within one slit to reduce. By default the longslit option assumes that the slit is spatially centered on the mask. If this assumption causes a wildly wrong initial guess for the wavelength range, it is possible to override this value by changing the \texttt{fuel.slits.range\_lambda0} variable, which is a 2-element array containing the minimum and maximum values to be considered for the wavelength of the first pixel.


% ---------------------------------------------------
\subsection{\texttt{flame\_cutouts}}

Using the slit definition from the previous step, the slits are extracted from each corrected frame and saved as FITS files. All the cutouts corresponding to the same slit are saved in the directory slitxx, where xx is the slit number.

% ---------------------------------------------------
\subsection{\texttt{flame\_spatialcal}}

The spatial rectification is calculated starting from the slit edges, and is corrected for nodding and dithering using the results of the diagnostics. The coefficients of the coordinate transformation $\gamma(x,y)$ are stored for each cutout in the fuel.slits structure.


% ---------------------------------------------------
\subsection{\texttt{flame\_roughwavecal}}

The wavelength calibration is arguably the most important and difficult step in the data reduction. An accurate calibration is not only useful for the scientific interpretation of the data, but is also necessary for a correct sky subtraction.

In \flame, the wavelength calibration is split into three steps. The first step finds an approximate wavelength calibration for the central part of the slit, using only the first science frame. The sky spectrum is extracted from the central five pixels of the slit, and is compared to a model spectrum. The comparison consists of two successive steps:
\begin{enumerate}
\item Both the observed and the model sky spectra are smoothed to an intermediate spectral resolution ($R=500$, but this value can be controlled via settings.roughwavecal\_R), and a series of values for the pixel scale (in micron per pixel) are tested. At each loop, cross-correlation is used to find the initial wavelength, and the pixel scale that gives the largest value of the cross-correlation is selected.
\item Once the pixel scale and the initial wavelength are roughly known, a finer comparison is performed. In this case the spectra are not heavily smoothed, so that the narrow OH lines can be used for a very effective comparison. The relation between $x$ pixel coordinate and wavelength is typically not linear, and a second-order polynomial is used: $\lambda(x) = a_0 + a_1 x + a_2 x^2$. A fine grid of values for $a_1$ and $a_2$ is used, and at each loop cross-correlation is used to determine the best value for $a_0$.
\item Finally, another cross-correlation is performed at the native spectral resolution.
\end{enumerate}

For each slit, plots illustrating the successive comparisons of observed and model sky spectra are saved as ps files (rough\_wavelength\_calibration.ps). An example is shown in Figure \ref{fig:wavecal_approx}.

If the sky spectrum includes a strong contribution from continuum emission, as is often the case for the thermal emission in the K band, it may be necessary to fit and subtract a low-order polynomial to the observed spectrum before comparing it to the model. This is controlled by the parameter settings.roughwavecal\_smooth\_window.


\begin{figure*}[tbp]
\begin{minipage}{\textwidth}
   \centering
 \includegraphics[width=0.31\textwidth]{wavecal1}
   \hspace*{0.01\textwidth}
 \includegraphics[width=0.31\textwidth]{wavecal2}
  \hspace*{0.01\textwidth}
 \includegraphics[width=0.31\textwidth]{wavecal3}
\end{minipage}
\caption{Approximate wavelength calibration. The sky spectrum (in black) is extracted from the central region of the slit, and is compared to a template spectrum (in red). The top panels show the initial guess, and the bottom panels show the results of the cross-correlation. The cross-correlation is repeated three times using successively higher spectral resolution.}
\label{fig:wavecal_approx}
\end{figure*}


% ---------------------------------------------------
\subsection{\texttt{flame\_findlines}}


The previous procedure gives only a rough wavelength calibration that is valid for the pixel rows at the center of the slits. Since the wavelength solution typically varies with spatial coordinate within the slit (particularly for tilted slits), the calibration needs to be calculated for each individual pixel row, which is done in this step.

Initially, the spectrum of the central pixel row is extracted, and the individual OH emission lines are identified using the rough wavelength solution. A Gaussian fit is performed on each line, and the $x$ pixel coordinate corresponding to the center is stored. It is therefore necessary that the calibration found in the previous step is accurate to within the window used in the Gaussian fit, otherwise the identification of the OH lines will not be possible. The relation between the $x$ coordinate and the expected wavelength of each OH line is then fit with a low order polynomial (see example in Figure \ref{fig:speclines}). This procedure is then repeated for the next pixel row, using the wavelength solution from the previous pixel row as first guess. This is always a good assumption as long as the wavelength solution varies smoothly with position, without abrupt changes from one pixel position to the next.

After all the pixel rows have been processed, the final result is a number of $x,y$ coordinates of detected OH lines, and the corresponding theoretical wavelength. A ds9 region file (OHlines.reg) showing the detected location of the OH lines is produced.

It is possible for difficult cases (e.g., few or weak lines, or slits heavily tilted), that the parameters controlling this step need to be tweaked (see Table \ref{tab:settings}).


\begin{figure}[tbp]
\centering
\includegraphics[width=0.7\textwidth]{wavecal_accurate}
\caption{Identification of sky lines. Top panel: The sky spectrum is extracted from a single pixel row, and using the approximate wavelength calibration the OH lines are identified and fit with a Gaussian profile. Bottom panel: the measured position of each OH line, together with their known wavelength, allow an accurate wavelength solution, shown in blue.}
\label{fig:speclines}
\end{figure}


% ---------------------------------------------------
\subsection{\texttt{flame\_wavecal}}

A 2D polynomial function is fit to the identified sky line to obtain the coordinate transformation $\lambda(x,y)$. The polynomial degrees of the function (both in x and y) can be changed for particularly tall or narrow slits.


% ---------------------------------------------------
\subsection{\texttt{flame\_illumcorr}}

The illumination correction is calculated from the sky lines and applied to each cutout. The user can choose to skip this step by setting the parameter illumination\_correction$=0$.


% ---------------------------------------------------
\subsection{\texttt{flame\_skysub}}

Because the OH emission lines vary on short timescales, the A-B subtraction typically leaves strong residuals. In order to improve the sky subtraction, \flame\ adopts the method of Kelson 2003. \flame\ performs this operation on each frame for every slit, and saves a FITS file of the sky-subtracted data. A graphic window shows, during this step, the sky emission spectrum (collapsed onto one dimension) and the B-spline fit for every frame. The user has ample control on the parameters controlling the b-spline model and the rejection of outliers, which may be need to be tweaked when the science targets take up most of the slit.


% ---------------------------------------------------
\subsection{\texttt{flame\_rectify}}

Using the 2D rectification coefficients calculated in \texttt{flame\_wavecal\_accurate}, this step performs both the wavelength calibration and the spatial rectification simultaneously, using only one interpolation. The interpolation method can be set to 'NaturalNeighbor', 'Linear', 'NearestNeighbor', or 'Quintic'.


% ---------------------------------------------------
\subsection{\texttt{flame\_combine}}

This step consists in the stacking of all the A and B frames. Sigma-clipping is first performed to remove cosmic rays and bad pixels, the default value is 2-sigma but this can be changed. The individual A and B stacks, and the difference A-B, are output as FITS file for each slit. Non-skysubtracted versions of the stacks are also saved. Also, an ABcombined FITS file is produced by summing A-B to the shifted B-A images, so that one positive central trace, corresponding to the total observing time, can be extracted for the science analysis.

If the dithering length, as calculated from the star traces, matches the distance between any two slits, then these are assumed to be matching slits and their A-B stacks are combined together. It is worth pointing out that this assumption breaks down if the star traces for the A and B positions actually belong to different stars.

By default, no weights are applied to the frames, but the user can set frame\_weights to 'None', 'Flux', 'Seeing', or 'Peak'.


% ---------------------------------------------------
\subsection{\texttt{flame\_checkdata}}

Finally, this module performs a quality check on the reduced data set. If present, the reference star spectrum is analyzed and the effective seeing calculated. The file reference\_star.ps is written, showing the spatial profile, the width and position of the stellar track as a function of wavelength, and the xtracted 1D spectrum of the star. Then, a ps file is written for each slit. The first plot shows the stacked sky spectrum, from which the sky lines are identified and fit, and the wavelength residuals and spectral resolution R are shown. Stats on the wavelength solution and the velocity resolution are shown at the bottom. The remaining plots illustrate the performance of the wavelength calibration as a function of wavelength or frame. Looking at these figures it is easy to see if one of more of the frames do not have a good wavelength solution. In such cases the user can decide to run the reduction again excluding those frames, or trying to improve the wavelength solution by tweaking the settings.




% ***********************************************************************************************
% ***********************************************************************************************
%									5. OTHER INSTRUMENTS
% ***********************************************************************************************
% ***********************************************************************************************

\section{Supported Instruments}
\label{sec:instruments}

All the \flame\ modules described in Section \ref{sec:data_reduction} are written in an instrument-independent way, and can be used on nearly any optical or near-infrared data set. The instrument-specific part is relegated to the initialization module. When running \flame, the correct initialization module must be called, and those options that are required and/or relevant to the specific instrument must be set. To simplify this procedure, a different driver file is available for each instrument, although in principle a skilled user should be able to easily write the correct driver for any supported instrument. All driver files are split into three parts:
\begin{enumerate}
  \item The inputs are set, in slightly different ways for different instruments.
  \item The specific initialization module is then run and the \fuel\ structure is created.
  \item The data reduction modules are called. This part is identical for all drivers.
\end{enumerate}

The list of files in the \texttt{flame/drivers/} directory shows what instruments are supported. Currently, this directory contains the following files:
\begin{itemize}
  \item[] \texttt{flame\_driver\_luci.pro}: driver for reducing data from the LUCI1 and LUCI2 instruments at the Large Binocular Telescope.
  \item[] \texttt{flame\_driver\_lris.pro}: driver for reducing data from the blue and red channel of LRIS at the Keck telescope.
  \item[] \texttt{flame\_driver\_complete.pro}: generic driver that contains all the possible options for the \input\ structure and for the settings in the \fuel\ structure; it can be used as a reference by copying the relevant lines and pasting them on the actual driver.
  \item[] \texttt{flame\_driver\_minimal.pro}: minimalistic driver that can be used to run the first two modules (diagnsotics and quick stack) on any data set, including instruments that are not supported.
\end{itemize}

We describe each supported instrument, and give indications on how to implement support for new instruments, in the remainder of this section.


\subsection{LUCI at LBT}

The reduction of LUCI data is described throughout Section \ref{sec:setup} and \ref{sec:data_reduction}, and no further discussion for this instrument is needed.


\subsection{LRIS at Keck}


\subsection{MOSFIRE at Keck}

\emph{Work in progress.}



\subsection{Diagnostics for data from unsupported instruments}
\label{sec:minimal}


\subsection{How to add support for a new instrument}











\end{document}
